\documentclass[11pt]{article}
\usepackage{amsmath, amssymb}
\usepackage[utf8]{inputenc}

% These packages are used for various math fonts. Url is for the bibliography
\usepackage{amsmath,amssymb,amsfonts,amsthm,xcolor,url}

% This is for making graphs
\usepackage{tikz}
% This is for citing code.
\usepackage{minted}
% This is a ham-fisted way to produce nice environments

\newtheorem{thm}{Theorem}
\newtheorem*{thm*}{Theorem}
\newtheorem{dfn}[thm]{Definition}
\newtheorem{lem}[thm]{Lemma}
\newtheorem{prop}[thm]{Proposition}
\newtheorem{cor}[thm]{Corollary}
\newtheorem{fact}[thm]{Fact}

\theoremstyle{definition}
	\newtheorem{ex}[thm]{Example}
	\newtheorem{exercise}{Exercise}
	\newtheorem{remark}{Remark}
	\newtheorem{question}[thm]{Question}
    \newtheorem{observation}{Observation}
    \newtheorem{thought}{Thought}
    \newtheorem{application}{Application}
	
\numberwithin{thm}{section}

\begin{document}

\title{Number Theory - Lecture 2 Handout}
%\author{Math Circle Summer 2018}

\maketitle

%%%%%%%%%%%%%%%%%%%%%%%%%%%%%%%%%%%%%%%
%%% Introduction to this document %%%%%
%%%%%%%%%%%%%%%%%%%%%%%%%%%%%%%%%%%%%%%
%\section{Lecture 1 Handout}

%%% 1.2.4 Exercises
\subsection{Exercises}

\begin{exercise} Is $0$ an even number or an odd number? Why?
\end{exercise}

\begin{exercise} Show that $109$ is odd using the definition of odd.
\end{exercise}

\begin{exercise} Is it true that if $x + y$ is even, then $x$ and $y$ are both integers?
\end{exercise}

\begin{exercise} Find all natural numbers $n$ such that $\sqrt{n}$ is irrational.
\end{exercise}

\begin{exercise} Is $\phi$, the Golden ratio, irrational? $\phi = \frac{1 + \sqrt{5}}{2}$.
\end{exercise}

\begin{exercise} Is $\sqrt[3]{2}$ an irrational number?
\end{exercise}

\begin{exercise} Prove that a number is rational iff it has a decimal representation that eventually repeats. For example, $\frac{1}{6} = 0.16666 \ldots$, and $\frac{1}{11} = 0.090909\ldots$ (Hint: Try the backwards direction first.)
\end{exercise}

\begin{exercise} Using the previous exercise (whether you have proved it or not), construct an irrational number by explicitly writing out its decimal representation.
\end{exercise}

\begin{exercise} Give an example of an $x^y$ that is rational, even though both $x$ and $y$ are irrational. You may use the fact that $\pi, e, \ln(2)$, and $\phi$ are all irrational.
\end{exercise}

\begin{exercise} There is a classic proof by ``excluded middle'' of the fact that there are irrational numbers $a,b$, such that $a^b$ is rational. Find it using $\sqrt{2}^{\sqrt{2}}$ and $\sqrt{2}$.
\end{exercise}

\begin{exercise} Archimedes' principle is the fact that for every $x > 0$ (no matter how small), and for every $N \in \mathbb{N}$ (no matter how large), there is an $n$ such that $x\cdot n > N$.

Use this to prove that for every two real numbers $a < b$, there is a rational number $\frac{p}{q}$ between $a$ and $b$. (Hint: First do this for $a=0$.)
\end{exercise}

\begin{exercise} Can $n!$ ever be a square number, with $n>1$?
\end{exercise}

\end{document}
