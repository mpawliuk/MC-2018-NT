\documentclass[11pt]{article}
\usepackage{amsmath, amssymb}
\usepackage[utf8]{inputenc}

% These packages are used for various math fonts. Url is for the bibliography
\usepackage{amsmath,amssymb,amsfonts,amsthm,xcolor,url}

% This is for making graphs
\usepackage{tikz}
% This is for citing code.
\usepackage{minted}
% This is a ham-fisted way to produce nice environments

\newtheorem{thm}{Theorem}
\newtheorem*{thm*}{Theorem}
\newtheorem{dfn}[thm]{Definition}
\newtheorem{lem}[thm]{Lemma}
\newtheorem{prop}[thm]{Proposition}
\newtheorem{cor}[thm]{Corollary}
\newtheorem{fact}[thm]{Fact}

\theoremstyle{definition}
	\newtheorem{ex}[thm]{Example}
	\newtheorem{exercise}{Exercise}
	\newtheorem{remark}{Remark}
	\newtheorem{question}[thm]{Question}
    \newtheorem{observation}{Observation}
    \newtheorem{thought}{Thought}
    \newtheorem{application}{Application}
	
\numberwithin{thm}{section}

\begin{document}

\title{Number Theory - Lecture 11 Handout}
%\author{Math Circle Summer 2018}

\maketitle

%%%%%%%%%%%%%%%%%%%%%%%%%%%%%%%%%%%%%%%
%%% Introduction to this document %%%%%
%%%%%%%%%%%%%%%%%%%%%%%%%%%%%%%%%%%%%%%

Messages from last class: Rohan 3323, Devanshu 2597, Max+Sophia 235-563, Tara+Lizette 732, Toril 64553, Vivek 31115, Caroline 2171-821, Drake 1816 

\begin{exercise} Find a subset of $51, 57, 34, 25, 46, 41, 25, 56, 64, 70$ with sum $151$.
\end{exercise}

\begin{exercise} Make a password with someone in the class using the Diffie-Hellman key exchange. Any information you agree on must be written on the board.
\end{exercise}

\begin{exercise} Find the smallest superincreasing sequence possible that starts with $1$ and has 10 terms. Assume that the sum of the first $n$ terms is greater or equal to the sum of the first $n-1$ terms.  
\end{exercise}

\begin{exercise} Create a challenging subset sum problem for someone in the class \ldots and solve someone else's subset sum problem.
\end{exercise}

\begin{exercise} Show Mike's Folly for $\mathbb{Z}_11$ by finding an element with order $10$, one with order $5$, and one with order $2$.
\end{exercise}

\begin{exercise} Explore Mike's Folly by finding the elements of $\mathbb{Z}_p$ that have order $\phi(n)$. Check this for various primes $p$.
\end{exercise}

\begin{exercise} (Vishnu's house of morbid curiosities and grotesque logic.) Mike's folly can be incredibly false. Show that in $\mathbb{Z}_{24}$ all its elements have order $1$ or $2$, but $\phi(n) = 8$. $\mathbb{Z}_{80}$ and $\mathbb{Z}_{120}$ also have very low maximum order, but large $\phi(n)$.
\end{exercise}
\end{document}
