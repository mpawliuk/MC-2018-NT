\documentclass[11pt]{article}
\usepackage{amsmath, amssymb}
\usepackage[utf8]{inputenc}

% These packages are used for various math fonts. Url is for the bibliography
\usepackage{amsmath,amssymb,amsfonts,amsthm,xcolor,url}

% This is for making graphs
\usepackage{tikz}
% This is for citing code.
\usepackage{minted}
% This is a ham-fisted way to produce nice environments

\newtheorem{thm}{Theorem}
\newtheorem*{thm*}{Theorem}
\newtheorem{dfn}[thm]{Definition}
\newtheorem{lem}[thm]{Lemma}
\newtheorem{prop}[thm]{Proposition}
\newtheorem{cor}[thm]{Corollary}
\newtheorem{fact}[thm]{Fact}

\theoremstyle{definition}
	\newtheorem{ex}[thm]{Example}
	\newtheorem{exercise}{Exercise}
	\newtheorem{remark}{Remark}
	\newtheorem{question}[thm]{Question}
    \newtheorem{observation}{Observation}
    \newtheorem{thought}{Thought}
    \newtheorem{application}{Application}
	
\numberwithin{thm}{section}

\begin{document}

\title{Number Theory - Lecture 7 Handout}
%\author{Math Circle Summer 2018}

\maketitle

%%%%%%%%%%%%%%%%%%%%%%%%%%%%%%%%%%%%%%%
%%% Introduction to this document %%%%%
%%%%%%%%%%%%%%%%%%%%%%%%%%%%%%%%%%%%%%%

\begin{exercise} If January 1 of this year was a Monday, what day of the week will next year start with?
\end{exercise}

\begin{exercise} What is the final digit of of $3^{2018}$?
\end{exercise}

\begin{exercise} What is the final digit of of $4^{2018}$?
\end{exercise}

\begin{exercise} What is the final digit of of $5^{2018}$?
\end{exercise}

\begin{exercise} What is the final digit of of $6^{2018}$?
\end{exercise}

\begin{exercise} What is the final digit of of $i^{2018}$? where $0\leq i < 10$?
\end{exercise}

\begin{exercise} Can a perfect square number end in $3$? What are the possible ending digits of a perfect square? If you claim that a perfect square can end in the digit $i$, you must give an example where this happens.
\end{exercise}

\begin{exercise} What are the possible ending digits of a number $n^4$? What about $n^8$? Make a conjecture.
\end{exercise}

\begin{exercise} Is $111 \cdots 1114$ ever equal to an $n^4$?
\end{exercise}

\begin{exercise} What are the possible conjugacy classes of $n^3 \mod 8$?
\end{exercise}

\begin{exercise} Show that the polynomial $x^6 - x^2 + 8x - 10$ cannot have any \textit{integer} solutions by showing that it has no solutions congruent to $0$ mod 4, or 1 or 2 or 3 mod 4.
\end{exercise}

\begin{exercise} For an odd number $n$, define $n!! = n(n-2)(n-4)\cdots(5)(3)(1)$. Find the final digit of $2017!!$.
\end{exercise}

\begin{exercise} Show that $2x \equiv 7 \mod 10$ has no solutions.
\end{exercise}

\begin{exercise} Use Bezout's Identity to show that the equation $ax \equiv 1 \mod n$ always has a solution if $\gcd(a,n) = 1$, (treat $a$ like a constant). This says that $a$ has a multiplicative inverse mod $n$.
\end{exercise}

\begin{exercise} Prove that equivalence mod $n$ is an equivalence relation. That is, show:
\begin{itemize}
	\item $a \equiv a \mod n$ for each $a$,
    \item If $a \equiv b \mod n$, then $b \equiv a \mod n$,
    \item If $a \equiv b \mod n$ and $b \equiv c \mod n$, then $a \equiv c \mod n$.
\end{itemize}
\end{exercise}

\begin{exercise} (Vishnu's house of morbid curiosities and grotesque logic.) What is $2018 \mod 1$? What about $2018 \mod 0$?
\end{exercise}

\begin{exercise} Can someone please explain the ``Doomsday Rule'' to me? See \url{https://en.wikipedia.org/wiki/Doomsday_rule}.
\end{exercise}

\begin{exercise} (Canada National Olympiad 2003) What are the final three digits of $2003^{2002^{2001}}$?
\end{exercise}

\end{document}
