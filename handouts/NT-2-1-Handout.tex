\documentclass[11pt]{article}
\usepackage{amsmath, amssymb}
\usepackage[utf8]{inputenc}

% These packages are used for various math fonts. Url is for the bibliography
\usepackage{amsmath,amssymb,amsfonts,amsthm,xcolor,url}

% This is for making graphs
\usepackage{tikz}
% This is for citing code.
\usepackage{minted}
% This is a ham-fisted way to produce nice environments

\newtheorem{thm}{Theorem}
\newtheorem*{thm*}{Theorem}
\newtheorem{dfn}[thm]{Definition}
\newtheorem{lem}[thm]{Lemma}
\newtheorem{prop}[thm]{Proposition}
\newtheorem{cor}[thm]{Corollary}
\newtheorem{fact}[thm]{Fact}

\theoremstyle{definition}
	\newtheorem{ex}[thm]{Example}
	\newtheorem{exercise}{Exercise}
	\newtheorem{remark}{Remark}
	\newtheorem{question}[thm]{Question}
    \newtheorem{observation}{Observation}
    \newtheorem{thought}{Thought}
    \newtheorem{application}{Application}
	
\numberwithin{thm}{section}

\begin{document}

\title{Number Theory - Lecture 5 Handout}
%\author{Math Circle Summer 2018}

\maketitle

%%%%%%%%%%%%%%%%%%%%%%%%%%%%%%%%%%%%%%%
%%% Introduction to this document %%%%%
%%%%%%%%%%%%%%%%%%%%%%%%%%%%%%%%%%%%%%%

\begin{exercise} Find the GCD of $a = 124$ and $b = 32$, by using the GCD algorithm. Use that to find a solution to $ax + by = \text{GCD}(a,b)$.
\end{exercise}

\begin{exercise} The GCD algorithm takes a long time if you use two adjacent Fibonacci numbers. Check this for $a = 34$ and $b = 55$. How many steps did it take? Can you generalize this result?
\end{exercise}

\begin{exercise} Does the GCD algorithm always output a remainder of $0$ after finitely many steps, or can it run forever? Explain your reasoning.
\end{exercise}

\begin{exercise} Why does the GCD algorithm work? In other words, why is the final remainder always the GCD of the numbers we started with? (Hint: Look at the GCD of consecutive remainders in the proof.)
\end{exercise}

\begin{exercise} Suppose that you have found a pairs of numbers $x,y$ that satisfy $ax + by = d$, and $d >0$. What relationship does $d$ have to $a$ and $b$? What happens in the special case of $d = 1$?
\end{exercise}

\begin{exercise} For any $a,n>0$ with $\textbf{GCD}(a, n) = 1$, show that the Euclidean algorithm can be used to find an $x$ such that $ax - 1$ is divisible by $n$. How does this change if $\textbf{GCD}(a, n) \neq 1$?
\end{exercise}

\begin{exercise} What is the GCD of consecutive positive numbers? Prove your statement.
\end{exercise}

\begin{exercise} What is the GCD of consecutive Fibonacci numbers? Prove your statement.
\end{exercise}

\begin{exercise} Can you predict which of the Fibonacci numbers will be even? (Gather data, make a hypothesis, try to prove it.) Can you find a more general statement?
\end{exercise}

\begin{exercise} Find a formula for the sum of the first $n$ Fibonacci numbers. 
\end{exercise}

\begin{exercise} There is only one square Fibonacci number larger than $1$. Find it.
\end{exercise}

\begin{exercise} Find a formula for the sum of the first $n$ \textit{odd-index} Fibonacci numbers.
\end{exercise}

\begin{exercise} Find a formula for the sum of the first $n$ \textit{even-index} Fibonacci numbers.
\end{exercise}

\begin{exercise} What is the GCD of any two Fibonacci numbers $F_n$ and $F_m$? Make a table of data.
\end{exercise}

\begin{exercise} The following data has been gathered from Fibonacci numbers. Find a pattern, or a special subset of numbers.

\begin{table}[ht!]
\begin{tabular}{l|l|l}
$x$ & $5x^2+4$ & $5x^2 - 4$ \\ \hline
1 & 9 & 1 \\
2 & 24 & 16 \\
3 & 49 & 41 \\
5 & 129 & 121 \\
8 & 324 & 316 \\
13 & 849 & 841 \\
21 & 2209 & 2201
\end{tabular}
\end{table}
\end{exercise}

\begin{exercise} (Putnam 2000 - B2) Show that for $n \geq m \geq 1$ the expression
\[
	\frac{\text{GCD}(m,n)}{n}\binom{n}{m}
\]
is always an integer. Recall $\binom{n}{m} = \frac{n!}{(n-m)!m!}$.
\end{exercise}
\end{document}
