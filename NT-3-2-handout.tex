\documentclass[11pt]{article}
\usepackage{amsmath, amssymb}
\usepackage[utf8]{inputenc}

% These packages are used for various math fonts. Url is for the bibliography
\usepackage{amsmath,amssymb,amsfonts,amsthm,xcolor,url}

% This is for making graphs
\usepackage{tikz}
% This is for citing code.
\usepackage{minted}
% This is a ham-fisted way to produce nice environments

\newtheorem{thm}{Theorem}
\newtheorem*{thm*}{Theorem}
\newtheorem{dfn}[thm]{Definition}
\newtheorem{lem}[thm]{Lemma}
\newtheorem{prop}[thm]{Proposition}
\newtheorem{cor}[thm]{Corollary}
\newtheorem{fact}[thm]{Fact}

\theoremstyle{definition}
	\newtheorem{ex}[thm]{Example}
	\newtheorem{exercise}{Exercise}
	\newtheorem{remark}{Remark}
	\newtheorem{question}[thm]{Question}
    \newtheorem{observation}{Observation}
    \newtheorem{thought}{Thought}
    \newtheorem{application}{Application}
	
\numberwithin{thm}{section}

\begin{document}

\title{Number Theory - Lecture 10 Handout}
%\author{Math Circle Summer 2018}

\maketitle

%%%%%%%%%%%%%%%%%%%%%%%%%%%%%%%%%%%%%%%
%%% Introduction to this document %%%%%
%%%%%%%%%%%%%%%%%%%%%%%%%%%%%%%%%%%%%%%

\begin{table}[!ht]
\begin{tabular}{l|l|l|l}
$n$ & $\phi(n)$ & $n$ & $\phi(n)$ \\ \hline
1   & 1         & 13  & 12        \\
2   & 1         & 14  & 6         \\
3   & 2         & 15  & 8         \\
4   & 2         & 16  & 8         \\
5   & 4         & 17  & 16        \\
6   & 2         & 18  & 6         \\
7   & 6         & 19  & 18        \\
8   & 4         & 20  & 4         \\
9   & 6         & 21  & 12        \\
10  & 4         & 22  & 10        \\
11  & 10        & 23  & 22        \\
12  & 4         & 24  & 8        
\end{tabular}
\end{table}

\begin{exercise} Find the value of $\phi(30)$ in two ways: (1) use the formula for $\phi(n)$, and (2) directly find the numbers between $1$ and $30$ that are relatively prime to $30$.
\end{exercise}

\begin{exercise} Compute $\phi(360)$ (\textit{no scope}).
\end{exercise}

\begin{exercise} What is the value of $ 20^{12} \mod 13$?
\end{exercise}

\begin{exercise} Using primes between $27$ and $100$ (or larger if you really want), set up your own public key. Put it on the board for people to send you messages. 
\end{exercise}

\begin{exercise} You may have noticed that all the $\phi(n)$ are even, except $\phi(1)$. Why is this? When is $\phi(n)$ divisible by $4$? Make a more general statement.
\end{exercise}

\begin{exercise} Suppose that $x$ divides $y$.  How do $\phi(x)$ and $\phi(y)$ relate to each other? Gather data, make a hypothesis, test it, refine it, then prove it.
\end{exercise}

\begin{exercise} Determine a pattern for
\[
	f(n) = \sum_{d | n} \phi(d)
\]
where the sum is taken over all divisors of $n$ (other than $1$). For example, $f(6) = \phi(1) + \phi(2) + \phi(3) + \phi(6)$. Prove it for any special cases you can, like primes, the product of two primes, or powers of primes.
\end{exercise}

\begin{exercise} Determine a pattern for
\[
	f(n) = \sum_{k, \gcd(k,n) = 1} k
\]
where the sum is taken over all numbers less than $n$, that are relatively prime with $n$. For example, $g(10) = 1 + 3 + 7 + 9$. Prove it for any special cases you can, like primes, the product of two primes, or powers of primes.
\end{exercise}

\begin{exercise} Caroline has sent you the message $3011$ using the public key from class $(N=5183, e=2033)$. How do you feel? (You may use the computer at the front of class to access Python to make computations.)
\end{exercise}

\begin{exercise} You have been eavesdropping on a conversation between Joy and Ethan during one of Mike's lectures. You receive the message $666807$, and you know they used the public key $(N=945543, e=180103)$. How should Mike feel?
\end{exercise}

\begin{exercise} \textbf{Try this again with your new knowledge about $\phi$.} (Canada National Olympiad 2003) What are the final three digits of $2003^{2002^{2001}}$?
\end{exercise}

\begin{exercise} (Vishnu's house of morbid curiosities and grotesque logic.) What would happen if you found an RSA public key where the modulus $N$ had over 23 million digits? Is this super secure, or super weak?
\end{exercise}
\end{document}
