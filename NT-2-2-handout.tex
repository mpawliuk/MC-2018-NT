\documentclass[11pt]{article}
\usepackage{amsmath, amssymb}
\usepackage[utf8]{inputenc}

% These packages are used for various math fonts. Url is for the bibliography
\usepackage{amsmath,amssymb,amsfonts,amsthm,xcolor,url}

% This is for making graphs
\usepackage{tikz}
% This is for citing code.
\usepackage{minted}
% This is a ham-fisted way to produce nice environments

\newtheorem{thm}{Theorem}
\newtheorem*{thm*}{Theorem}
\newtheorem{dfn}[thm]{Definition}
\newtheorem{lem}[thm]{Lemma}
\newtheorem{prop}[thm]{Proposition}
\newtheorem{cor}[thm]{Corollary}
\newtheorem{fact}[thm]{Fact}

\theoremstyle{definition}
	\newtheorem{ex}[thm]{Example}
	\newtheorem{exercise}{Exercise}
	\newtheorem{remark}{Remark}
	\newtheorem{question}[thm]{Question}
    \newtheorem{observation}{Observation}
    \newtheorem{thought}{Thought}
    \newtheorem{application}{Application}
	
\numberwithin{thm}{section}

\begin{document}

\title{Number Theory - Lecture 6 Handout}
%\author{Math Circle Summer 2018}

\maketitle

%%%%%%%%%%%%%%%%%%%%%%%%%%%%%%%%%%%%%%%
%%% Introduction to this document %%%%%
%%%%%%%%%%%%%%%%%%%%%%%%%%%%%%%%%%%%%%%

\begin{exercise} Write down the general set of solutions to $84x + 35y = 7$. Recall that yesterday we found a solution to this equation.
\end{exercise}

\begin{exercise} \label{ex:bens} How many ways are there to make \$100 by only using \$10 bills and \$20 bills? To solve this, set up a linear Diophantine equation and find its full set of solutions. Then decide how many of those solutions give reasonable physical interpretations in this question.
\end{exercise}

\begin{exercise} You are stranded on an island without a calculator, but lots of paper and working pens. Describe in a step-by-step manner how you would find all solutions to $2019x + 93y = 11$.
\end{exercise}

\begin{exercise} You are stranded on an island without a calculator, but lots of paper and working pens. Describe in a step-by-step manner how you would find all solutions to $2019x + 93y = 12$.
\end{exercise}

\begin{exercise} Show that if $\gcd(a,c)=1$ and $\gcd(b,c)=1$ then $\gcd(ab,c) = 1$. (Hint: Use Bezout's identity.)
\end{exercise}

\begin{exercise} (Related to exercise \ref{ex:bens}.) Make a conjecture about the number of \textit{positive} integer solutions (both $x,y>0$) an equation $ax+by=n$ can have when $a,b,n >0$. Prove it. 
\end{exercise}

\begin{exercise} Make a conjecture about the number of \textit{mixed} integer solutions ($x$ and $y$ have opposite signs) an equation $ax+by=n$ can have when $a,b,n >0$. Prove it. 
\end{exercise}

\newpage

The next three exercises prove that our theorem about the solutions of a linear Diophantine equation is correct.

\begin{exercise} Show that if $x_0, y_0$ is a solution to $ax+by=n$, then every number of the form
\[
	\left(x_0 + m\frac{b}{\gcd(a,b)}, y_0 + m\frac{a}{\gcd(a,b)} \right)
\]
is also a solution to $ax+by=n$.
\end{exercise}

\begin{exercise} (You will need this for the next exercise.) Show that $\frac{a}{\gcd(a,b)}$ and $\frac{b}{\gcd(a,b)}$ have $\gcd$ 1.
\end{exercise}

\begin{exercise} Show that if $x_0, y_0$ and $x_1, y_1$ are both solutions to $ax+by=n$, then
\[
	\frac{a}{\gcd(a,b)}(x_0 - x_1) = -\frac{b}{\gcd(a,b)}(y_0 - y_1)
\]
and so by the previous exercise, there must be an integer $m$ such that $(x_0 - x_1) = m \frac{b}{\gcd(a,b)}$ and $y_0-y_1 = -m \frac{b}{\gcd(a,b)}$.
\end{exercise}

\begin{exercise} (The chicken McNugget problem) Chicken McNuggets come in packages of 3, 9 or 20. We know that since $\gcd(3,20) = 1$ every combination at or above $(20-1)(3-1) = 38$ can be made. What are the impossible combinations between $1$ and $37$?
\end{exercise}

\begin{exercise} (AIME 1994 Problem 11) Ninety-four bricks, each measuring $4''\times10''\times19'',$ are to be stacked one on top of another to form a tower 94 bricks tall. Each brick can be oriented so it contribues $4''\,$ or $10''\,$ or $19''\,$ to the total height of the tower. How many different tower heights can be achieved using all ninety-four of the bricks?
\end{exercise}

\end{document}
